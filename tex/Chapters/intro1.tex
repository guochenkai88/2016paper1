\section{Introduction}
%\baigd{Topic: Popularity of Android and event driven feature in mobile device}
Android has proved a prevalent mobile operating system installed on contemporary smartphone and tablet --- it has been dominating the market with more than 80\% share~\cite{idc}. 
As a result, numerous Android apps have been extensively used in the users' daily lives and even in many security-sensitive scenarios. 
For a long time, the data confidentiality of Android apps has been recognized to be critical and has been extensively studied~\cite{flowdroid,....@chenkai,find 4 here}. 
However, the other aspect about the confidentiality of an app, namely the ``control'' aspect, 
has not drawn symmetrical attention. 
The control indicates the program location that is being executed at a particular time point, 
and we call this \emph{control confidentiality}. 
Violation of the control confidentiality enables the attacker to observe the execution state of the victim app, 
and in consequence, the attacker becomes able to mount fine-grained attacks such as tapjacking~\cite{@chenkai: add citation here, you can find the citation in the oakland paper}, 
clickjacking~\cite{find one or two, may be GUI state paper mentioned some} and GUI confusion~\cite{oakland paper}, at a precise timing. 

Android is a ``multitasking'' OS that execute multiple apps from multiple origins in a concurrent way, and  
%Each app is executed within a isolated sandbox, with the objective that an app, 
%even if it is compromised, cannot interfere the data and code of another app. 
each app is executed within a isolated sandbox. 
This sandbox mechanism, as stated in the Android documentation\footnote{http://developer.android.com/training/articles/security-tips.html}, aims to \emph{``isolate(s) your app data and code execution from other apps''}. 
This vague statement seems to imply that the sandbox mechanism 
provides an app with confidentiality and integrity protection over both of its data and control.  
However, what is demonstrated by related work~(and this work) actually contradicts this implication ---
the design of the Android OS does not take control confidentiality into consideration. 
For example, Ren et al.~\cite{usenix15 paper} and Bianchi et al.~\cite{oakland paper} 
exploits the insecure design of the task management to launch task hijacking and GUI confusion. 
%Prior studies have contributed their efforts to push this boundary. 
Chen et al.~\cite{GUI state paper} infers the victim app's GUI states from a malware through shared memory to conduct right-time GUI hijacking. 
%\baigd{@chenkai: mention related papers here. make sure to contain all the major ones.}
In this paper, we introduce a novel type of attacks which exploit another insecure design in Android---the unique execution model of Android apps.

%The popularity of Android can be partially attributed to its decent support of rich ambience interaction~(such as interaction with user and sensors) and GUI expressiveness, 
%which are desirable in modern mobile devices. 
Android uses a decent execution model to support rich ambience interaction~(such as interaction with user and sensors) and GUI expressiveness, 
which are desirable in modern mobile devices. 
%The way that Android uses to implement its interaction 
This execution model is the so-called \emph{event-driven} and \emph{callback} mechanism, 
which is fundamentally different from the traditional execution model like the single-entry-point model of desktop programs. 
In particular, a program running on Android changes its control flow according to the occurrence of certain events, instead of in a pre-defined execution order. 
This mechanism provides an asynchronous way for a specific app to invoke its certain methods~(i.e., event handlers) when a certain event happens. 
As a typical example, a pre-defined \texttt{onClick} callback function in an app is invoked and only invoked after the certain registered button is clicked by the user. 

The event-driven mechanism, however, sometimes reveals control information of an app to another malicious app on the device. 
This threat occurs when an event triggers the execution of the callbacks from different components and even different apps. 
We call this kind of events \emph{public events}  
and corresponding callbacks \emph{public event callbacks}~(PEC). 
The public events are ubiquitous in Android. Examples 
include system events, implicit intents, service running status, etc. 
Observing the occurrence of a public event, 
a malicious app can learn that in each of the apps which have registered the corresponding event callbacks, 
the control is transferred to the corresponding PEC. 
If the PEC leads to some program locations where the attacker plans to mount his attacks, 
the attacker successfully learns a precise timing to commit his attacks. 
A typical example is to exploit the PEC for GUI attacks~\cite{chen2014peeking,bianchi2015app,ren2015towards} in the popular social app \texttt{WeChat}. 
It contains a ``Shake for assigned chats'' module, such that 
a random online friend is recommended after the user slightly shake the device. 
However, the shake action is a public event. 
A malicious app running in the background can observed this action, so that it can judge the timing of certain foreground display by \texttt{WeChat}. 
Further, the malware is able to craft attacks like phishing and spoofing exploiting the judgement to the foreground. 
In general, PEC violates the control confidentiality and enables the attacker to learn a precise timing. 
%The PEC is insensitive on surface but offers attack materials for other apps with evil purpose. 
In section 4, we present in detail how such threats can be involved into various types of attacks. 


To the best of our knowledge, there has not a previous work exclusively focusing on analysing and tackling the issues arising by the PEC threats. In this paper, we first engage a comprehensive analysis and a large scale of market apps survey about the PEC. Fortunately, Android ecosystem provider a relatively open environment allowing one to make a pre-analysis of a targeted app off-line. Such feature makes it possible to identify the potential victim apps suffering from the PEC threats. Nonetheless, the identification to the PEC within a certain app should overcome two main challenges. One is that attackers have to obtain materials of targeted apps on the code level overcoming the code confusion. The other one is to detect out at least one feasible inferring path across numerous control flow paths within the targeted app. Such path should contain a callback function that can be invoked by an exposed public event and a display-sink(e.g., startActivity or Toast.makeText) point with exposure features. For all, it is entirely possible to identify a victim for a wily attacker. 

%The other challenge is that it should be detected that existing a determinable path from the start of such callback function to the display-sink, which involves fine-grain control flow analysis. To achieve this goal, branch conditions and method invocation have to be properly handled to avoid the restriction of permission request and user-interaction.

Based on above consideration, it has a substantial value to conduct a previous evaluation for current common-used market apps. We implement a prototype static analyzer CSDroid with 4k lines python codes to automatically conduct the PEC identification to 2375** apps collected from Googly Play. Of all the targeted apps, we find 783** such sensitive paths within 354** targeted apps, including some popular apps like **. Moreover, it's surprising that 69\% **  of apps with service component suffer from such threats, which alerts us a high risk of such potential security threats within background components widely existing in generic apps.  

%Given such an app with certain PEC weaknesses, an adversary attacker could devise a rich variety of attacks. Since such attacks are mainly based on the display-sink, the attack types are somehow similar with traditional UI attacks. However, as the display-sink should be reasonably inferred from the PEC, adversary app needs to concern more details about time-delay and program flow conditions. Besides that, traditional UI attacks mainly rely on activity display of victim app; display-sink contains more types of displays such as dialog and notification, which allow attackers to flexibly craft more attacks contents. To validate the feasibility of such PEC weaknesses exploitation, we explore and implement a set of proof-of-concept attacks based on such threats, including not only traditional phishing and spoofing to sensitive pages, but also newly dialog and notification exploitation for privilege escalation and other privacy steal. 

To migrate the impact of such PEC threats, we also discuss corresponding defence strategy. 

To summarize, the main contribution of this paper are:

1. We systematically study vulnerabilities of Android app lead by abuse of public event callbacks as well as discover various {\color{red}four} types of attack vectors leading to such PEC threats. 
 
2. By a set of corresponding proof-of-concept attacks, we validate the feasibility of such PEC exploitation and illustrate the severity of such threats. 

3. We introduce a tool implemented to automatically detect the PEC threats within targeted apps. By our study, a wide range of apps are affected by PEC attacks.

4. We provide a mitigation strategy to the PEC threats, under which a more security callback invoking mechanism can be established.

%On the other side, Android sandbox is known as an effect mechanism to protect apps from other one's attack. Sandbox restricts the resources to be offered for certain independent entities, leaving only very few ways for them to communicate with others. Actually, through the usage of conventional so-called side-channels, certain indirect inferring gradually evolves into a typical sort of attacks. In general, this sort of attacks threat a broad range of apps causing side-channels(such as process status) exist in system level. In other words, most of developed apps would be unable to avoid the influence brought by features from system level. Nevertheless, the impact appears limited for the deviation posed by sophisticated algorithms from current side-channel approaches. For instance, \cite{chen2014peeking}(UI state) treats several selected process statuses as side-channels, and utilizes them to infer a proper emergence time for the fake phishing Activity devised by attackers. However, the critical inference bases on a series of data collection and computations, leading to an inevitable derivation in term of device and targeted app itself. A typical example is that the typical Activity transition signal presence in \cite{chen2014peeking} is quite week and the shape is also changed in our reproducing experiment. Thus, using the same procedures as proposed in \cite{chen2014peeking} would hardly perform an ideal attack. The similar situation exists in other existing side-channel attacks aiming to Android apps as well \cite{zhou2013identity}\cite{jana2012memento}.
