%
\section{Background} 
%
In order to facilitate the understanding of the PEC weakness, this section presents an introduction to the event-driven and callback mechanism in Android. 

%Based on that, we define the public events and public event callbacks. 

%In order to facilitate the understanding of the \atk, 
%this section introduces some relevant background on the event-driven and callback mechanism in Android. 

%\subsection{Events and Callbacks}
In Android, events and callbacks are the ways that an app interacts with the Android system and users. 
In this paper, we define an \textit{event} as a situation that can trigger certain program logic to work. 
For instance, the explicit intent broadcast is an event that can trigger the apps receiving it to execute specific code.
%In addition, we define a \textit{public event} as an event that can trigger the 
%program logics from \emph{more than one apps} to work. 
%Correspondingly, an implicit intent broadcast caters the definition of public event since it can be received by multiple apps. 
%Apparently, the public event is the subset of the event.
In addition, we define a \textit{event callback} as the code~(e.g., a function and a program branch) that is executed after certain event(s) happen(s), 
and the execution is only related to such event(s). For example, an \texttt{onClick} method would only be executed after user clicks a certain button. 

\subsection{Events in Android}
According to the sources that events originated from, the events in Android can be categorized into three groups, 
namely \emph{life-cycle} event, \emph{user-driven} event and \emph{system-driven} event. 
%As listed in Table~\ref{tab:events}, each category contains public events that may violate the host apps' control confidentiality. 
In the following, we briefly introduce the events in each category. 

\begin{itemize}
    \item \textbf{Life-cycle events.} An app contains multiple \emph{life-cycle} states, and the state is changed as a user launches, pauses and resumes the app. 
When a new state is entered, the corresponding callbacks will be invoked. 
%Android apps spontaneously arouse corresponding callback functions according to the life-cycle step it resides. We call the event that can trigger life-cycle method to execute as life-cycle event. 
For instance, an Activity executes its \texttt{onCreate} function when it is initially loaded, and executes the \texttt{onPause} callback after it loses the screen focus. 
%Life-cycle callbacks are somehow executed 
%along with user's interactions and impacted by users. 

%On surface, due to constraint of sandbox mechanism, the life-cycle phrases (Create,Pause,Destroy, etc.) of an given activity seem to be isolated from components belong to other apps. 
%However, the event triggering certain life-cycle callbacks like onCreate and onDestroy enables other app to predict it in an indirect way. For example, the event that an activity is launching leads to the onCreate method in that activity executing. We conclude such life-cycle event that can expose the execution trace of certain app in Table 1.

%The trigger-events originated by different sources are equipped with different trigger features. For instance, a \textit{low-battery} event would happen only when the rest quantity of battery becomes as low as a pre-defined level, which is irrelevant to user's interaction. We categorize the events lied in Android that can lead to certain invocation of callback as three groups, namely \textit{life-cycle} event, \textit{user-driven} event, and \textit{system-driven} event.

    \item \textbf{User-initiated events. }
        As interacting with an app, the user is guided to perform diverse actions on screen, 
        %User regularly are guided to perform diverse actions on screen, 
        such as tapping, dropping up and down, click and double-click. 
        Right after capturing an action from users, Android system would react following the corresponding callback function that reflects the functionality and logic designed by developer. 
        %Generally, a user-driven event triggers the code execution of the one getting screen focus, which is hard capture by other apps. However, there still exists some user-driven events unrelated to the screen(e.g., shake action) acting as public events. 
    \item \textbf{System events.} System events normally come from the situation that a given system status changes. 
        In particular, when the system events happen, Android OS broadcast a message or directly invokes the callback functions defined by the apps. 
For example, when the memory of the device becomes lower than a pre-defined value, 
the OS invokes \texttt{onLowMemory()} if any app has pre-implemented it. 

%If not specified, system event could be obtained by any app in theory so long as one is granted proper permissions. Generally, the permission granting could hardly attract user's attention for system event such as battery status is insensitive to user. To the end, system event provides an indirect entry for adversary app getting parts of the running process of other apps that implement system-driven callbacks interfaces. Table1 lists parts of system event instances.
    
\end{itemize}


\begin{table*}[t]
\centering
 \caption{\label{tab:events}Parts of events and callbacks in Android}
 \begin{tabularx}{\linewidth}{XXXXX} 
  \toprule
  Categories & Event & Callback & Permission & Public \\
  \midrule
  life-cycle & app starting & onCreate/onStart(launching activity); statical broadcast receiver & N/A & Y\\
             & activity losing focus & onPause & N/A & N\\
             & service starting & onCreate/onBind & N/A & Y\\
             
  user-driven & clicking button & onClick & N/A & N\\
              & shaking & onSensorChanged & N/A & Y\\
              
  system  & Intent.ACTION\_BATTERY\_LOW & onReceive & N/A & Y \\
          & Intent.ACTION\_BOOT\_COMPLETED & onReceive & RECEIVE\_BOOT\_COMPLETED & Y \\
  
  \bottomrule
 \end{tabularx}
\end{table*}

\subsection{Callbacks and Components}
%Android platform is a kind of open source operating system established by Linux core. 
Android allows the developers to design and develop their own apps in a flexible way, 
%causing Android freely 
by providing a variety of open-source libraries and frameworks for developers. 
In Android, four types of components plays the key role in app's construction, namely \textit{Activity}, \textit{Service}, \textit{Broadcast Receiver} and \textit{Content Provider}. 
\textit{Activity} is designed as a graphical interface for interacting with users. 
\textit{Service} is mainly used for handling complex computing job in background without a graphical user interface. 
\textit{Broadcast Receiver} provides a global listener that could trigger specific code logic when receive relative message. 
\textit{Content Provider} providers a universal interface for data storage and management, and for sharing data between different components. 
%As a side effect of 
In order to implement the user-driven mechanism, Android provides numerous callback interfaces to be implemented by the developers. 
Most of them rely on certain component, which we give details in follows.

\begin{itemize}
    \item \textbf{Callbacks in Activity.}  
        While all three types of callbacks can be included by an Activity component,             
        most of the callbacks in it is user-driven events, given that 
        Activity is mainly used for providing a visible graphic interfaces so as to interact with users. 
        

        %As aforementioned, upon the protection of sandbox, these callback statuses are hardly obtained by other apps only if the event triggered by user is irrelevant to screen focus. For instance, a shake action from users would lead to \textit{sensor} related events, which have no especial association with the screen. Besides, some system-driven callbacks are implemented in activity for catering the requirements of functionalities. 

    \item \textbf{Callbacks in Service.} 
        Handling program logic and computing tasks in background, Service has no direct interaction with users, which is typically different from Activity. 
        By this means, most callbacks in a Service component deals with the life-cycle and system events. 
        It is also possible to trigger a callback through a user-driven event in service, which is irrelevant to the visible screen~(e.g., shaking the phone to trigger a \texttt{SensorChanged} event). 

%Without specified receiver, this kind of events could be easily obtained by other apps either.   
        

%Yet, user's action without screen focus could be an exception, which we will discuss later.

%Service life-cycle consists of five callback functions(onCreate, onStartCommand, onBind, onUnbind and onDestroy) but has a simpler life-cycle procedure compared with Activity. Being the same as Activity, these life-cycle callbacks are isolated from components of other apps so that has little utility value for attacker. 

%Regularly, most of \textit{Services} are designed to wait for a typical signal and then invoke corresponding callback functions to push their computing result to an output interface, such as a \textit{dialog} or an \textit{Activity}. In general, the signal could be any type of event, invocation from other component or a program condition. If the trigger event belongs to system events, it can be easily obtained by component from other apps. This makes possible for attackers to perceive the execution process of a target app.

%It is also possible to trigger a callback through a user-driven event in service, which is irrelevant to visible screen (e.g., shaking the phone to trigger SensorChanged event). Without specified receiver, this kind of events could be easily obtained by other apps either.   

    \item \textbf{Callbacks in Broadcast Receiver.} 
        The \textit{Broadcast} in Android has a wide spectrum of sources, from system to other components and can be delivered across all the installed apps through a common medium named \textit{Intent}. 
        Generally, callback functions in Broadcast Receiver only act as a channel that responses the coming Broadcast and then launches corresponding \textit{service} or \textit{activity} to handle it. 
    However, once lacking of protection, a Broadcast Receiver could suffer from the threats of repeatedly receiving trash messages from an adversary Broadcast sender. 
    Similarly, it is also possible that a typical Broadcast is intercepted by a deliberately designed Receiver. 
    Therefore, developers normally add some protection countermeasures, such as setting the \texttt{export} property as false or using self-defined permission. 

    \item \textbf{Callback in Content Provider. }
        A Content Provider component acts as a channel for operating database in Android system, so callbacks in Content Provider appear quite trivial. 
        Similar to Service, a Content Provider component is designed as a background component and 
        its lifecycle callback \texttt{onCreate()} is mainly used to initialize the provider object. 
        Since this type of component is mostly used internally in an App, we do not consider the PEC in it. 

       % However, the provider is initiated when invoked by the operation from ContentResolver related to user-driven and program logic. 
       % Therefore, attackers are hard to leverage the process causing the launching time of a certain provider is hard to be captured by other apps. 
       % Besides, developers seldom set security related logics within \texttt{onCreate()} and other callbacks of Content Provider. 
       % Based on above consideration, we exclude the Content Provider in later PEC discussion.

\end{itemize}


